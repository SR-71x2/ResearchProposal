%!TEX root = ../../main.tex

\chapter{Forschungsmethodik und Vorgehen}
\section{Validierung der Modelle}
Im vorherigen Kapitel wurden anhand von internationalen Studien Verfahren für das PCR-Pooling erarbeitet.
Die ermittelten Modelle sollen in diesem Kapitel durch Forschungsmethoden der Wirtschaftsinformatik validiert werden.

Bei mangelhafter Umsetzung einer Pooling-Methode, könnten die Fehlerrate der Testverfahren steigen oder Proben kontaminiert werden.
Das Ergebnis wären ein Mehraufwand durch erneute Testung oder unentdeckte Fehldiagnosen.
Ziel ist es, eine geeignete Methode und Skalierung zu finden und diese auf Robustheit gegen Fehlern zu überprüfen.
Als Forschungsmethoden für die Validierung sind eine \textbf{Simulation} oder eine \textbf{argumentativ-deduktiven Analyse} vorgesehen.
Hierbei sollen die Grenzen der beschriebenen Verfahren erforscht werden.

Getestet werden sollen beispielsweise:
\begin{itemize}
	\setlength{\itemsep}{-8pt}
	\item Unterschiedliche Infektionswahrscheinlichkeiten in der Testgruppe
	\item Clusterbildung unter den Positivfällen
	\item Falschergebnis einzelner (Teil-)Testungen
	\item Kontaminierung der Proben
	\item Fehler bei der Probenvermischung
\end{itemize}

Aus den verfügbaren Pooling-Verfahren des vorherigen Kapitels soll so ein möglichst effizientes und robustes Modell gewählt werden.
Es sollen optimale Parameter für das Modell gefunden werden, um die in der Forschungsfrage formulierten Ziele zu erfüllen.

Die Ergebnisse der Validierung werden gegebenenfalls als Anpassungen in die Modellen eingearbeitet.
Dies soll die Grundlage für die Auswahl einer effizienten Methode sein, welche im Zuge der sekundären Forschungsfrage implementiert wird.

\section{Implementierung im betrieblichen Umfeld}
In diesem Kapitel soll eine Referenzimplementierung der erarbeiteten Modelle in das betriebliche Umfeld erstellt werden.
Hierbei handelt es sich um die sekundäre Forschungsfrage.
Die primären Forschung soll im Falle eines Ressourcenkonflikts priorisiert werden.\footnote{Vgl. Gliederung der Arbeit}

In der Medizin sowie im betrieblichen Umfeld, funktioniert Skalierung grundsätzlich anders als in der Informatik.
In der betrieblichen Umsetzung bedeutet die Verdopplung der Personenzahl einen massiven Mehraufwand bei Logistik und Organisation.
Um die ermittelte Effizienzsteigerung in der Praxis zu erreichen, müssen die Abläufe fehlerfrei ausgeführt werden.
Eine Aufgabe der Implementierung ist es, Risiken für operative Fehler minimieren.
Erreicht werden kann dies durch betriebliche Abläufe, Dokumentation und die Reduzierung der Arbeitsschritte.
Hierfür sollen praktische Empfehlungen gegeben werden.
Zudem wird die Logistik und entstehende Kosten der Testung beschrieben.

Dieses Kapitel wird sich auf einige Standardliteratur aus den Bereichen Prozess- und Ablauforganisation stützen.
Im Schwerpunkt handelt es sich hierbei allerdings um ein induktives Kapitel mit dem Ziel, einen Ausblick auf mögliche Implementierungsstrategien zu geben.
Eine abschließende Behandlung der betrieblichen Abläufe wird im Rahmen der Forschungsarbeit nicht möglich sein.
Aufgrund der Individualität jedes Unternehmens sollen allgemeingültige Empfehlungen gegeben werden.

\section{Zu erwartende Eigenbeiträge}
Die wissenschaftliche Lage verändert sich im Rahmen der Corona-Pandemie nahezu täglich und es ist schwierig, hierzu einen nachhaltigen Beitrag zu leisten.
Die vorgeschlagene Arbeit orientiert sich deshalb zwar am aktuellen Bedarf der pandemischen Lage - beschränkt sich jedoch nicht auf diesen.

Weder das PCR-Testverfahren noch die Idee zum Pooling von Proben ist neu in der COVID-19-Pandemie entstanden.
Diese Verfahren sind bewährt und werden auch nach der Pandemie noch zum Einsatz kommen.
Die Forschung nutzt die Pandemie somit als Anhaltspunkt und Praxisbeispiel, in der Hoffnung einen kurzfristigen Beitrag leisten zu können.
Die Ergebnisse sollen hierbei abstrahierbar für weitere Anwendungsfälle bleiben.