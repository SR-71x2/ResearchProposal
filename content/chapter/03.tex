%!TEX root = ../../main.tex

\chapter{Wissenschafltiche Methodik  / Überarbeitung offen}
\section{Forschungsmethoden und Quellen}
Das Kapitel Einleitung wird notwendigerweise eine sachliche, aber relativ unwissenschafltiche Beschreibung der tagesaktuellen Lage.
Hierbei werden Zeitungsartikel sowie aktuelle Gesetze und Verordnungen zum Einsatz kommen um der Arbeit einen Anker im ständig wechselnden Pandemiegeschehen zu geben.
Unterlegt werden diese Beschreibungen durch wissenschafltiche Artikel, welche allerdings aufgrund der Tagesaktualität teilweise von Pre-Print-Servern stammen werden und noch keinen Peer-Review erfahren haben.
Hierdurch wird es schwierig, den wissenschafltichen Anspruch einzuhalten, weswegen dieses Kapitel kritisch refelektiert werden muss und wissenschafltich schwierige Passagen auf das Einleitungskapitel beschränkt werden sollen.

\section{Wissenschaftlicher Mehrwert und Forschung}
Die wissenschafltiche Lage verändert sich im Rahmen der Corona-Pandemie nahezu täglich und es ist schwierig, hierzu einen nachhaltigen Beitrag einzubringen.
Die vorgeschlagene Arbeit orientiert sich deshalb zwar am aktuellen Bedarf der pandemischen Lage - beschränkt sich jedoch nicht auf diesen.
Neben den stark schwankenden Ereignissen der Pandemie werden fundierte mathematische Verfahren aus der Informatik genutzt und für medizinische Anwendungen übertragen.
Weder das PCR-Testverfahren noch die Möglichkeiten zur Fehlerkorrektur in Daten sind neu.
Diese Verfahren sind bewährt und werden auch nach der Pandemie noch eingesetzt werden.
Die Forschung nutzt die Corona-Pandemie somit als Anhaltspunkt und Praxisbeispiel, in der Hoffnung einen kurzfristigen Beitrag leisten zu können.
Die Ergebnisse sollen hierbei abstrahierbar für weitere Anwendungsfälle bleiben. \cite{Testbuch}