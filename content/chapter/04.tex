%!TEX root = ../../main.tex

\chapter{Gliederung der Arbeit}
\subsubsection{Einleitung}
Die Einleitung der Forschungsarbeit soll mit einer zeitgeschichtlich Verankerung innerhalb der Pandemie beginnen.
Diese Einordnung wird als notwendig erachtet, da sich die Situation in der Pandemie laufend verändert.
Die Lage kann sich durch das Aufkommen neuer Virusvarianten, neue Forschungserkenntnisse oder die verfügbarkeit neuer Impfstoffe verändern.
Die Bearbeitung der Forschungsarbeit wird hierdurch beeinflusst.
Gravierende Veränderungen können zukünftig zur Obsoleszenz der Ergebnisse führen.
Die Möglichkeit zur späteren Einordnung der Arbeit ist deshalb wichtig.
Hierfür muss die Lage zum Entstehungszeitpunkt der Arbeit kurz beschrieben werden.

Der Lagebeschreibung folgt eine Zielsetzung, welche die Forschungsfragen beschreibt.
Diese wird große Ähnlichkeit mit dem vorliegenden Research Proposal aufweisen.


\subsubsection{Ergebnis}
Die Arbeit endet mit einem Kapitel, in welchem die Erlebnisse zusammengefasst und Handlungsempfehlungen gegeben werden.
Es wird geprüft ob das Forschungsziel erreicht wurde und wie hoch das Optimierungspotenzial gegenüber den bisherigen Verfahren ist.
Auf dieser Basis wird ein Ausblick auf die Potenziale der betrieblichen Umsetzung gegeben.

Abhängig vom endgültigen Schwerpunkt der Arbeit, könnte die Implementierung der Verfahren ein Unterkapitel "Ausblick" im Rahmen des Ergebnisses werden.
Die Details der Implementierung würden hierdurch aus der Arbeit ausgelagert und abgegrenzt werden.
Sinnvoll könnte dies sein, wenn die primäre Forschungsfrage aufgrund von vielen unterschiedlichen Verfahren einen größeren Anteil der verfügbaren Seitenzahl beansprucht.
Die sekundäre Forschungsfrage wird somit flexibel angepasst, um ausreichend Ressourcen für die primäre Forschungsfrage bereitzustellen.