%!TEX root = ../../main.tex

\chapter{Einleitung}
\section{Problemstellung}
Der Einsatz von Tests ist eine der wichtigsten Maßnahmen zur Bekämpfung der Pandemie.
Die Sensitivität der Schnelltests ist allerdings nach aktueller Auffassung nicht ausreichend.
Die Uni Würzburg testete im XXXX 2021 die Produkte mehrerer Hersteller.
Das Ergebnis war eine durchschnittliche Sensitivität von 42,6 Prozent.\footnote{Q: Uni Würzburg}
Bei geimpften Personen ist die Sensitivität der Schnelltests aufgrund der niedrigeren Virenlast noch schlechter. \footnote{Q: Niedrigere Virenlast Geimpfte}
Zielführend wäre deshalb, einen größtmöglichen Bevölkerungsanteil regelmäßig mittels des PCR-Verfahrens zu testen.
Aktuell ist dies aufgrund der hierfür erforderlichen Laborkapazitäten und hohen Kosten nicht umsetzbar. \footnote{Q: Überlastung PCR}

TestCite \cite{dreyfus:1980}

\section{Zielsetzung}
Die vorgesehene Arbeit möchte Möglichkeiten erforschen, vorhandene PCR-Testkapazitäten effizienter zu allokieren.
Eine signifikante Erhöhung der Testquote und -qualität soll hierdurch erreicht werden.
Bestehende Algorithmen aus der Informatik sind geeignet, dies zu leisten.

Die Algorithmen sollen auf Ihre Tauglichkeit in der medizinischen Anwendung und auf ihre Anfälligkeit für menschliche Fehler untersucht werden.
Gesucht wird ein Algorithmus, welcher die folgenden Eigenschaften erfüllt:

\begin{itemize}
\item Signifikante Steigerung der Testkapazitäten
\item Senkung der Testkosten pro Patient
\item Robustheit gegen Fehlanwendung
\item Geringes Risiko von falsch-positiven Ergebnissen (ethische Bewertung)
\item Anwendbarkeit im betriebliche und medizinische Umfeld
\end{itemize}


% \acf{Polymerase chain reaction}
