%!TEX root = ../../main.tex

\chapter{Problemstellung und Forschungsziel}
Der Testung der Bevölkerung ist eine der wichtigsten Maßnahmen zur Pandemiebekämpfung.
Die Sensitivität\footnote{Quote der Infizierten, welche durch den Test korrekt identifiziert werden.}
der Schnelltests ist allerdings nach den bisherigen Erkenntnissen nicht ausreichend.\footcite{WuerzburgStudie}
Zielführend wäre deshalb, einen größtmöglichen Bevölkerungsanteil regelmäßig mittels des PCR-Verfahrens\footnote{\acf{Polymerase chain reaction}}
 zu testen.
Aktuell ist dies aufgrund der hierfür erforderlichen Laborkapazitäten und hohen Kosten nicht umsetzbar.\footnote{Q: Überlastung PCR}

Die vorgesehene Arbeit hat zum Ziel, Methoden zu erforschen um die vorhandene PCR-Testkapazitäten effizienter zu nutzen.
Erreicht werden soll dies, durch die Anwendung von Pooling-Verfahren.
Hierbei werden die Proben mehrerer Patienten kombiniert und gemeinsam getestet.
Die Kapazitäten für PCR-Tests sollen hierdurch gesteigert werden, um diese anstelle von Schnelltests einsetzen zu können.
Ziel ist, hierdurch eine Qualitätssteigerung bei der anlasslosen Massentestung zu erreichen.

Ziel der Arbeit ist, ein Testverfahren nach den folgenden Anforderungen zu identifizieren:
\begin{itemize}
\setlength{\itemsep}{-8pt}
\item Signifikante Steigerung der Testkapazitäten
\item Senkung der Testkosten pro Patient
\item Robustheit gegen Fehlanwendung
\item Geringes Risiko von falsch-positiven Ergebnissen
\item Anwendbarkeit im betrieblichen Umfeld
\end{itemize}

Während der Pandemie haben sich viele Forschergruppen an Pooling-Verfahren gearbeitet.
Hierbei sind viele unterschiedliche Ansätze entstanden.
Die vorliegende Arbeit möchte diese Methoden zunächst analysieren und zu Clustern ähnlicher Methoden aggregieren.
Anschließend soll  eine Überprüfung auf Effizienz und Robustheit der unterschiedlichen Verfahren stattfinden.
Hierfür sollen Forschungsmethoden aus der Wirtschaftsinformatik eingesetzt werden.
Eine Simulation ist vorgesehen oder eine argumentativ-deduktive Analyse.

Primäres Forschungsziel ist die Identifizierung eines effizienten und robusten Testverfahrens.
Sekundäres Forschungsziel ist die Erarbeitung einer Referenzimplementierung für das betriebliche Umfeld.