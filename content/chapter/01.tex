%!TEX root = ../../main.tex

\chapter{Problemstellung und Forschungsziel}
Der Testung der Bevölkerung ist eine der wichtigsten Maßnahmen zur Pandemiebekämpfung.
Zielführend wäre deshalb, hierbei die ungenauen Antigen-Schnelltests\footcite{WuerzburgStudie}
durch das präziseren PCR-Verfahren\footnote{Polymerase Chain Reaction}
zu ersetzen.
Aktuell ist dies aufgrund der hierfür erforderlichen Laborkapazitäten und hohen Kosten nicht umsetzbar.\footcite{rki-bericht_2021}

Die vorgesehene Arbeit hat zum Ziel, existierende Pooling-Verfahren zu überprüfen.
Die Proben mehrerer Patienten werden hierbei kombiniert und gemeinsam getestet.
Diese Verfahren bieten erhebliche Steigerungspotenziale für die Kapazitäten der PCR-Testungen.\footcite{Aertzeblatt}
Bei der anlasslosen Massentestung könnte hierdurch eine Qualitätssteigerung erreicht werden.
Identifiziert werden soll ein Pooling-Verfahren, welches diese Steigerung ohne signifikanten Qualitätsverlust ermöglicht.
Ziel der Arbeit ist deshalb, Pooling-Methoden nach den folgenden Anforderungen zu prüfen:
\begin{itemize}
\setlength{\itemsep}{-8pt}
\item Potenzial zur Erhöhung der Testkapazitäten
\item Robustheit gegen Fehlanwendung
\item Risiko von falschen Ergebnissen
\item Anwendbarkeit im betrieblichen Umfeld
\end{itemize}

Während der Pandemie haben sich viele Forschungsgruppen an Pooling-Verfahren gearbeitet.\footcite{viehweger_increased_2020}
Hierbei sind viele unterschiedliche Ansätze entstanden.\footcite{verwilt_evaluation_2021}
Die vorliegende Arbeit möchte diese Methoden zunächst analysieren und zu Clustern ähnlicher Verfahren aggregieren.
Anschließend soll  eine Überprüfung auf Effizienz und Robustheit der unterschiedlichen Verfahren stattfinden.
Hierfür sollen eine Simulation oder eine argumentativ-deduktive Analyse eingesetzt werden.

\textbf{Primäres Forschungsziel}\newline
 Analyse existierender PCR-Pooling-Verfahren.\newline
 Überprüfung dieser Methoden auf Effizienz und Robustheit.
 
 \textbf{Sekundäres Forschungsziel}\newline 
 Erarbeitung einer Referenzimplementierung für das betriebliche Umfeld.\newline