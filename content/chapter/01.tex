%!TEX root = ../../main.tex

\chapter{Einleitung}
\section{Problemstellung}
Der Testung der Bevölkerung ist eine der wichtigsten Maßnahmen zur Pandemiebekämpfung.
Die Sensitivität
\footnote{Quote der Infizierten, welche der Test korrekt identifiziert.}
der Schnelltests ist allerdings nach den bisherigen Erkenntnissen nicht ausreichend.
\footcite{Wuerzburg-Studie}
Zielführend wäre deshalb, einen größtmöglichen Bevölkerungsanteil regelmäßig mittels des PCR-Verfahrens
\footnote{\acf{Polymerase chain reaction}}
 zu testen.
Aktuell ist dies aufgrund der hierfür erforderlichen Laborkapazitäten und hohen Kosten nicht umsetzbar.
 \footnote{Q: Überlastung PCR}

\section{Zielsetzung}
Die vorgesehene Arbeit hat zum Ziel, Methoden zu erforschen um die vorhandene PCR-Testkapazitäten effizienter zu nutzen.
Erreicht werden soll dies, durch die Anwendung von Pooling-Verfahren.
Hierbei werden die Proben mehrerer Patienten kombiniert und gemeinsam getestet.
Die Kapazitäten für PCR-Tests sollen hierdurch gesteigert werden, um diese anstelle von Schnelltests einsetzen zu können.
Ziel ist, hierdurch eine Qualitätssteigerung bei der anlasslosen Massentestung zu erreichen.

Gesucht wird eine Methode, welcher die folgenden Eigenschaften erfüllt:

\begin{itemize}
\setlength{\itemsep}{-8pt}
\item Signifikante Steigerung der Testkapazitäten
\item Senkung der Testkosten pro Patient
\item Robustheit gegen Fehlanwendung
\item Geringes Risiko von falsch-positiven Ergebnissen
\item Anwendbarkeit im betrieblichen Umfeld
\end{itemize}

Forschungsziel dieser Arbeit ist die Identifizierung einer geeigneten PCR-Pooling-Methode und ihre Implementierung im betrieblichen Umfeld.