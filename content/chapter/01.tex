%!TEX root = ../../main.tex

\chapter{Einleitung}
\section{Problemstellung}
Der Einsatz von Tests ist eine der wichtigsten Maßnahmen zur Bekämpfung der Pandemie.
Die Sensitivität der Schnelltests ist allerdings nach aktueller Auffassung nicht ausreichend.
Die Uni Würzburg testete im XXXX 2021 die Produkte mehrerer Hersteller.
Das Ergebnis war eine durchschnittliche Sensitivität von 42,6 Prozent.
\footcite{Wuerzburg-Studie}
Bei geimpften Personen ist die Sensitivität der Schnelltests aufgrund der niedrigeren Virenlast noch schlechter.
 \footnote{Q: Niedrigere Virenlast Geimpfte}
Zielführend wäre deshalb, einen größtmöglichen Bevölkerungsanteil regelmäßig mittels des PCR-Verfahrens
\footnote{\acf{Polymerase chain reaction}}
 zu testen.
Aktuell ist dies aufgrund der hierfür erforderlichen Laborkapazitäten und hohen Kosten nicht umsetzbar.
 \footnote{Q: Überlastung PCR}

\section{Zielsetzung}
Die vorgesehene Arbeit hat zum Ziel, Methoden zu erforschen um die vorhandene PCR-Testkapazitäten effizienter zu nutzen.
Die Anzahl der möglichen PCR-Tests soll hierdurch gesteigert werden, um eine Qualitätssteigerung in der Testung zu erreichen.

Bestehende Verfahren sollen analysiert und miteinander vergleichen werden.
Gesucht wird eine Methode, welcher die folgenden Eigenschaften erfüllt:

\begin{itemize}
\item Signifikante Steigerung der Testkapazitäten
\item Senkung der Testkosten pro Patient
\item Robustheit gegen Fehlanwendung
\item Geringes Risiko von falsch-positiven Ergebnissen (ethische Bewertung)
\item Anwendbarkeit im betriebliche und medizinische Umfeld
\end{itemize}

