%!TEX root = ../../main.tex

\chapter{Einleitung}
\section{Problemstellung}
Der Testung der Bevölkerung ist eine der wichtigsten Maßnahmen zur Pandemiebekämpfung.
Die Sensitivität
\footnote{Quote der Infizierten, welche der Test korrekt identifiziert.}
der Schnelltests ist allerdings nach aktuellen Erkenntnissen nicht ausreichend.
Die Uni Würzburg testete im XXXX 2021 die Produkte mehrerer Hersteller.
Das Ergebnis war eine durchschnittliche Sensitivität von 42,6 Prozent.
\footcite{Wuerzburg-Studie}
Bei geimpften Personen ist die Sensitivität der Schnelltests aufgrund der niedrigeren Virenlast noch schlechter.
 \footnote{Q: Niedrigere Virenlast Geimpfte}
Zielführend wäre deshalb, einen größtmöglichen Bevölkerungsanteil regelmäßig mittels des PCR-Verfahrens
\footnote{\acf{Polymerase chain reaction}}
 zu testen.
Aktuell ist dies aufgrund der hierfür erforderlichen Laborkapazitäten und hohen Kosten nicht umsetzbar.
 \footnote{Q: Überlastung PCR}

\section{Zielsetzung}
Die vorgesehene Arbeit hat zum Ziel, Methoden zu erforschen um die vorhandene PCR-Testkapazitäten effizienter zu nutzen.
Erreicht werden soll dies, durch die anwendung von Pooling-Verfahren.
Hierbei werden die Proben mehrerer Patienten kombiniert und gemeinsam getestet.
Die Kapazitäten für PCR-Tests sollen hierdurch gesteigert werden, um diese anstelle von Schnelltests einzusetzen.
Ziel ist, hierdurch eine Qualitätssteigerung in der anlasslosen Massentestung zu erreichen.

Bestehende Verfahren sollen analysiert und miteinander vergleichen werden. 
Im Laufe der Arbeit soll hierfür ein Kriterienkatalog entwickelt werden.


\section{Anforderungen an das Verfahren (ggf späteres Kapitel / Forschungsfrage)}
Gesucht wird eine Methode, welcher die folgenden Eigenschaften erfüllt:

\begin{itemize}
\item Signifikante Steigerung der Testkapazitäten
\item Senkung der Testkosten pro Patient
\item Robustheit gegen Fehlanwendung
\item Geringes Risiko von falsch-positiven Ergebnissen
\item Anwendbarkeit im betrieblichen und medizinischen Umfeld
\end{itemize}

