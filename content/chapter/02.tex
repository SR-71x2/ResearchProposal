%!TEX root = ../../main.tex

\chapter{Geplanter Aufbau der Arbeit}
\section{Einleitung}
Die Einleitung soll aus einer Problemstellung und einer Zielsetzung bestehen.
Die Problemstellung wird einen ungewöhnlich großen Umfang einnehmen müssen, um die Forschungsarbeit zeitgeschichtlich in der Pandemie einzuordnen.
Die Pandemielage, sowie ergriffene Maßnahmen, wissenschaftliche Erkenntnisse und die öffentliche Meinung verändern sich wöchentlich.
Hierdurch wird es notwendig, die Lage zum Entstehungszeitpunkt der Arbeit einzuordnen und zu beschreiben.
Veränderungen während des Erstellungszeitraums sind zu erwarten, wodurch voraussichtlich eine Beschreibung der Entwicklung erforderlich wird.
Die Zielsetzung beschreibt die Forschungsfragen und das Ziel der Arbeit.
Sie wird sich in großen Teilen am vorliegenden Research Proposal orientieren.

\section{Covid19-Testung}
Das ersten Hauptkapitel soll die aktuellen Erkenntnisse zu den Tests auf eine Covid19-Infektion beschreiben.
Zentrale Themen werden die Genauigkeit und Fehlerquoten der Tests sein (Sensitivität/Spezifität).
Im Folgenden werden die Möglichkeiten erörtert, einen Test für mehrere Personen zu verwenden und welchen Effekt dies nach aktueller Erkenntnis auf die Fehleranfälligkeit der Tests hat.
Grundlegende Prämisse der Arbeit ist, dass es möglich ist die Proben mehrerer Personen für einen PCR-Test zu kombinieren und gemeinsam auszuwerten.
Die Praxis an Schulen sowie das Funktionsprinzip eines PCR-Tests legen diese Möglichkeit nahe.

\section{Bestehende PCR-Pooling-Verfahren}
Im Laufe der Pandemie wurden von vielen Forschungsgruppen und Laboren Methoden entwickelt, um PCR-Pooling durchzuführen.
Die Skalierung ist hier sehr unterschiedlich.
Es gibt widersprüchliche Aussagen dazu, wie robust das PCR-Verfahren gegen Verwässerung der Proben ist.
Einige behaupten man könne maximal 5 Personen gemeinsam testen. \cite{Quelle}
Andere testen 25-40 Personen gemeinsam. \cite{Quelle}

In Deutschland haben lt. dem Ärzteblatt die Blutspendedienste am meisten Erfahrung mit PCR-Pooling-Verfahren. \cite{Ärzteblatt}
Seit Jahrzehnten kommen hier Pooling-Verfahren zum EInsatz, um Blutspenden auf HIV und Hepatitis zu testen.
Die Blutspendedienste haben hierfür auch ein Patent angemeldet. \cite{Patent Blutspende}
Die Methode dieses Patents wird als aktuellen Stand in Deutschland angenommen und soll die Basis für den Vergleich anderer Verfahren sein.

\section{Fehlerkorrektur-Programmierung (ECC) in der Informatik}
In der Informatik stellt sich bereits seit langem das Problem, Fehler in Speichern und bei Signalübertragung zu erkennen und zu korrigieren.
Dies wird über unterschiedliche - dem Anwendungsfall angepasste - Paritätsalgorithmen erreicht.
Ziel hierbei ist immer, den Overhead durch die Parität gering zu halten und zeitgleich die Integrität der Daten sicherzustellen.
Einhergehend damit muss eine Abwägung getroffen werden, in welchem Umfang Fehler und Datenverlust akzeptabel sind.

Der Hamming Code war einer der ersten ECC-Algorithmen und wurde in den 1950ern von Richard W. Hamming entwickelt.
Seine Verteilung der Paritäts-Bits ermöglicht eine effiziente Überprüfung der Daten.
Durch Verwendung von N+1 Paritäts-Bits, kann die Integrität von 2-hoch-N Daten zu überprüfen  werden.
Die Berichtigung einzelner Bitfehler ist ebenfalls möglich.
Sollte mehr als ein Fehler innerhalb des Blocks auftreten, kann dieser zwar erkannt, aber nicht berichtigt werden.

Für einen Speicherbereich mit 256 Bit werden somit 8+1 Paritäts-Bits benötigt, was einem Overhead von nur 3,5 Prozent entspricht.
Neuere Algorithmen haben die Effizient weiter gesteigert und sollen im Laufe der Arbeit vergleichen werden.

\section{Modellerstellung}
Geprüft werden soll die Übertragbarkeit mehrerer in der Informatik gängigen ECC-Algorithmen auf den medizinischen Bereich.
Die Theorie wäre, dass Covid-Infektionen bei anlasslosen Testungen wie auch Bitfehler selten sind.
Somit könnten dieselben Algorithmen zur Effizienzsteigerung genutzt werden.
Ziel ist es, eine Möglichkeit zu finden die exponentielle Effizienzsteigerung von ECC-Algorithmen auf medizinische Testungen anzupassen und hierdurch die Kosten deutlich zu reduzieren.
Hierbei müssen Anpassungen an den Algorithmen vorgenommen werden und neue Herausforderungen beachtet werden.

Beispielsweise sind bei einem 15-11-Hamming-Code nur 11/16 Bits echte Daten.
Die Paritätsbits kosten hier direkt Speicherkapazität.
Bei einer PCR-Testung wären theoretisch alle 16 Plätze verwendbar, da die Durchführung von PCR-Tests selbst (anders als bei Speicher) keine Plätze kostet.
Desweiteren sind neue Probleme zu erwarten, wenn die Tests von Menschen durchgeführt werden.
Hierbei entstehen Fehler, welche in den bisherigen Algorithmen keine Berücksichtigung finden mussten.


\section{Betriebswirtschaftliche Erwägungen}
In diesem Kapitel sollen betriebswirtschaftliche Aspekte Beachtung finden, insbesondere Skalierung und Logistik.
Bei der Informatik, sinkt der Paritätsbedarf bei einer erhöhten Datenmenge.
Hier ist in der Theorie eine endlose Steigerung der Datenmenge effizient, bis zu einem Punkt an dem mehr Bitfehler wahrscheinlich sind als das System berichtigen kann.
In der Medizin und im betrieblichen Einsatz funktioniert Skalierung grundsätzlich anders.
Hier bedeutet die Verdopplung der Personenzahl eine massiven Mehraufwand bei Logistik und Organisation.
Außerdem sind menschliche Fehler wahrscheinlicher und führen bereits bei geringeren Gruppengrößen zu einem wahrscheinlichen Versagen des Systems.
Es soll auf Basis der Kenntnisse aus dem Algorithmen-Kapitel mehrere Berechnungsmethoden und Skalierungsfaktoren erarbeitet werden.

\section{Simulation}
Es sollen andere Verfahren aus internationalen Studien herausgearbeitet und neue anhand Methoden aus der Informatik (Codingtheorie) erarbeitet werden. 

Die ermittelten Modelle sollen simuliert werden, um die Skalierung und Auswirkung von Fehlern zu überprüfen.
Hierdurch sollen optimale Parameter für das Modell anhand der in der Forschungsfrage formulierten Ziele ermittelt werden.
Ziel ist die Auswirkung von unterschiedlichen Arten und Wahrscheinlichkeiten eines Anwendungsfehlers oder eines fehlerhaften Tests zu ermitteln.
Dies soll bei der Auswahl eines effizienten Algoritmus unterstützen.

Verglichen werden diese dann auf Effizienz und Robustheit durch eine Simulation.
Diese ist vorgesehen, indem die Methoden in Software nachgebaut werden und zufällig generierte Testszenarien durchlaufen.
Hierbei sollen Kombinationen gesucht werden, in denen das Modell ein unerwünschtes Ergebnis liefert.
Ein Beispiel hierfür wäre die Ausstellung eines falschen Negativzertifikats.

Die Ergebnisse der Simulation sollen als Grundlage genutzt werden, um gegebenenfalls Anpassungen an den Modellen vorzunehmen.
Hierbei soll ein möglichst effizientes und robustes Modell entstehen und geprüft werden, ob Optimierungspotenzial gegenüber dem bisherigen Verfahren der Blutspendedienste existiert.

\section{Ergebnis}
Platzhalter
