%!TEX root = ../../main.tex

\chapter{Aktueller Stand der Wissenschaft}
\section{Das PCR-Testverfahren}
Um die Grundlagen für spätere Kapitel zu legen, soll zunächst die allgemeine Funktionsweise des PCR-Verfahrens erläutert werden.
Die verfahrensüblichen Qualiätsmerkmale Sensitivität und Spezifität werden beschrieben und auf Ablauf sowie Logistik der Testungen eingegangen.

Das PCR-Verfahren existierte bereits vor der COVID-19-Pandemie.
Es wird seit 1983
\footcite{wink_pcr_1994}
zur Erkennung von Viruserkrankungen eingesetzt.
Sowohl wissenschaftliche Literatur\footcite{schmidt_novel_2020}
als auch praxisnahe Publikationen\footcite{wehrle_weber_pcr_1994}
sind verfügbar.\footcite{clewley_polymerase_1995}
Forschungsrelevante Aussagen zur Wirksamkeit und Fehlerquote der Testung werden hierbei ausschließlich auf Quellen gestützt, welche die wissenschaftlichen Qualitätsstandards erfüllen.
Für betriebswirtschaftliche und ablauforganisatorische Themenbereiche wird die Einbeziehung von praxisnahe Literatur als sinnvoll erachtet.

\section{PCR-Pooling-Verfahren}
PCR-Pooling-Verfahren wurden bereits vor der Pandemie zur Diagnose anderen Viruserkrankungen erfolgreich eingesetzt.
\footcite{Aertzeblatt}
Hierbei werden die Proben mehrerer Patienten vermischt, um durch einen gemeinsamen Test den Aufwand zu senken.
Im Laufe der Pandemie wurden von vielen Forschungsgruppen und Laboren Methoden entwickelt, um PCR-Pooling durchzuführen.
\footcite{calabrese_how_2021}
Zur Robustheit des PCR-Verfahren gegen Verwässerung der Proben und den Skalierungsmöglichkeiten gibt es widersprüchliche Aussagen.
Einige Methoden empfehlen, dass maximal fünf Personen gemeinsam getestet werden.
\footcite{schmidt_novel_2020}
Andere vermischen die Proben von 25-40 Patienten.
\footcite{verwilt_evaluation_2021}
Das Ärzteblatt bescheinigt den Blutspendediensten die meiste Erfahrung mit PCR-Pooling-Verfahren.
\footcite{Aertzeblatt}
Um Blutspenden auf HIV und Hepatitis zu testen, kommen Pooling-Verfahren hier seit Jahrzehnten zum Einsatz.

Ein Vergleich dieser Pooling-Studien und der erforschten Methoden wird ein Schwerpunkt dieses Kapitels.
Ein Fokus ist hierbei die Fehleranfälligkeit der Tests bei unterschiedlichen Bedingungen und Verfahren.

Die wissenschaftliche Artikel, welche den Verfahren zugrunde liegen, werden aufgrund der Tagesaktualität teilweise von Pre-Print-Servern stammen.\footcite{viehweger_increased_2020}
In diesen Fällen ist noch keinen Peer-Review erfolgt, weswegen diese Quellen besonders kritisch reflektiert und mit anderen Publikationen verglichen werden müssen.\footcite{verwilt_evaluation_2021}
Die Einhaltung des wissenschaftlichen Anspruchs wird durch den Vergleich der vielen Publikationen und auf Basis der zugrundeliegenden Literatur sichergestellt.
Eine Validierung und Plausibilisierung der Methoden ist ohnehin im Rahmen der primären Forschungsfrage vorgesehen.

\section{Methoden der Kanalcodierung}
Ein Forschungsgebiet der Informatik ist die Integritätsprüfung von Speichern und Signalübertragungen.
Die Forscher entwickeln Algorithmen, für die Erkennung und Berichtigung von Fehlern.\footcite{hamming_information_1987}
Die Anforderungen sind hierbei stark abhängig vom Anwendungsfall und der zu erwartenden Fehlerverteilung.\footcite{blahut_algebraic_1992}

Dieses Kapitel dient dazu, ein Verständnis für die Funktionsweise unterschiedlicher Codierungsverfahren zu vermitteln.
Anhand von Beispielen werden die Unterschiede und Eigenschaften verschiedener Verfahren aufgezeigt.
Verdeutlicht wird hierdurch, nach welchen Kriterien Codierungsverfahren bewertet und verglichen werden können.
Aus der Kanalcodierung werden Anforderungen, Konzepte und Terminologie\footcite{dankmeier_codierung_1994}
für die Analyse der Pooling-Verfahren übernommen.

Die Existenz dieses Kapitels ist noch ungewiss.
Sein Zweck wäre, die theoretische Grundlage für eine argumentativ-deduktive Analyse der Pooling-Methoden zu liefern.
Sollte diese Analyse zugunsten einer Simulation entfallen, ist dieses Kapitel obsolet.\footnote{Vgl. Gliederung der Arbeit}