%!TEX root = ../../main.tex

\chapter{Geplanter Aufbau der Arbeit}
\section{Einleitung}
Die Einleitung der Forschungsarbeit soll mit einer zeitgeschichtlich Einordnung in der Pandemie beginnen.
Diese Einordnung wird als notwendig erachtet, da sich die Situation in der Pandemie laufend verändert.
Veränderungen - wie das Aufkommen neuer Virusvarianten - sind in der Lage, die Bearbeitung der Forschungsarbeit zu beeinflussen und zur Obsoleszenz ihrer Ergebnisse zu führen.
Die Lage zum Entstehungszeitpunkt der Arbeit soll deshalb beschrieben werden.
Eine spätere Einordnung der Arbeit wird hierdurch ermöglicht.

Der Lagebeschreibung folgt eine Zielsetzung, welche die Forschungsfragen und das Ziel der Arbeit beschreibt.
Sie wird sich in großen Teilen am vorliegenden Research Proposal orientieren.

\section{Verfahren zur Covid19-Testung}
\subsection{Das PCR-Testverfahren}
Das ersten Hauptkapitel soll die aktuellen Erkenntnisse und Verfahren zur Testung auf eine Covid19-Infektion beschreiben.
Die allgemeine Funktionsweise des PCR-Verfahrens soll hierbei den Einstieg bilden.
Zentrale Themen werden die Genauigkeit und Fehlerquoten der Tests sein.
\footnote{Bemessung anhand der verfahrensüblichen Qualiätsmerkmale Sensitivität und Spezifität}
Der Ablauf und die Logistik von Covid19-Testungen wird hierbei beleuchtet.

Das PCR-Verfahren existierte bereits vor der Covid19-Pandemie.
Es wird seit den 19XXern
\footcite{Quelle erster Masseneinsatz PCR}
zur Erkennung von Viruserkrankungen eingesetzt.
Sowohl wissenschaftliche Literatur
\footcite{Wissenschaftlich PCR}
als auch praxisnahe Publikationen
\footcite{Praxishandbuch PCR}
sind verfügbar.
Forschungsrelevante Aussagen zur Wirksamkeit und Fehlerquote der Testung werden hierbei ausschließlich auf Quellen gestützt, welche wissenschaftliche Qualitätsstandards erfüllen.
Für betriebswirtschaftliche und ablauforganisatorische Themenbereiche wird die Einbeziehung von praxisnahe Literatur als sinnvoll erachtet.

\subsection{Methoden der Kanalcodierung}
Die Integritätsprüfung von Speichern und Signalübertragungen stellt ein großes Forschungsgebiet der Informatik dar.
Die Kanalcodierung beschäftigt sich hierbei mit der Entwicklung von Algorithmen für die Erkennung und Berichtigung von Fehlern.
Die Anforderungen sind hierbei stark abhängig vom Anwendungsfall und der zu erwartenden Fehlerverteilung.

Ziel hierbei ist immer, den Overhead durch die Parität gering zu halten und zeitgleich die Integrität der Daten sicherzustellen.
Einhergehend damit muss eine Abwägung getroffen werden, in welchem Umfang Fehler und Datenverlust akzeptabel sind.

Ziel dieses Kapitels ist es, ein Verständnis für die Funktionsweise unterschiedlicher Codierungsverfahren zu gewinnen.
Es sollen anhand von Beispielen die Unterschiede und Eigenschaften verschiedener Verfahren aufgezeigt werden.
Verdeutlicht wird hierdurch, nach welchen Kriterien Codierungsverfahren bewertet und verglichen werden können.

Im Zusammenhang dieser Arbeit, sollen aus der Kanalcodierung Anforderungen, Konzepte und Terminologie übernommen werden.
Basierend auf diesen Erkenntnissen sollen Rahmenbedingungen definiert werden, welche für eine Covid19-Testung akzeptabel sind.
Diese sollen im folgenden Kapitel als Rahmenbedingungen für die Optimierung genutzt werden.

Eine direkte Verwendung bekannter Fehlerkorrektur-Algorithmen in der Medizintechnik wird an dieser Stelle geprüft.
Die Möglichkeit einer direkten Implementierung ist allerdings aufgrund sehr unterschiedlicher Abläufe unwahrscheinlich.
Hier soll in der Arbeit eine Abgrenzung erfolgen.

\subsection{PCR-Pooling-Verfahren}
Basierend auf den Erkenntnissen über das PCR-verfahren und die Methodiken der Kanalcodierung, soll in diesem Kapitel der Transfer erfolgen und Verfahren betrachtet werden, einen Test für mehrere Personen zu verwenden.
Diese sogenannten Pooling-Verfahren wurden bereits bei anderen Viruserkrankungen erfolgreich eingesetzt.
\footcite{Ärzteblatt}
Im Laufe der Pandemie wurden von vielen Forschungsgruppen und Laboren Methoden entwickelt, um PCR-Pooling durchzuführen.
\footcite{Reddit Quelle}
Die Skalierungen sind hier sehr unterschiedlich.
\footcite{Alternative Quelle Pooling}
Zur Robustheit des PCR-Verfahren gegen Verwässerung der Proben gibt es widersprüchliche Aussagen.
Einige behaupten man könne maximal 5 Personen gemeinsam testen.
\footcite{Quelle}
Andere testen 25-40 Personen gemeinsam.
\footcite{Quelle}
In Deutschland haben lt. dem Ärzteblatt die Blutspendedienste am meisten Erfahrung mit PCR-Pooling-Verfahren.
\footcite{Ärzteblatt}
Seit Jahrzehnten kommen hier Pooling-Verfahren zum Einsatz, um Blutspenden auf HIV und Hepatitis zu testen.
Die Blutspendedienste haben hierfür auch ein Patent angemeldet.
\footcite{Patent Blutspende}
Die Methode dieses Patents soll als Basis für den Vergleich anderer Verfahren verwendet werden.

Der Vergleich dieser Studien und die Beschreibung der Methoden soll ein Schwerpunkt dieses Kapitels werden.
Auf die Fehleranfälligkeit der Tests unter unterschiedlichen Bedingungen wird ein weiteres Augenmerk gelegt.

Unterlegt werden diese Beschreibungen durch wissenschaftliche Artikel, welche allerdings aufgrund der Tagesaktualität teilweise von Pre-Print-Servern stammen werden und noch keinen Peer-Review erfahren haben.
Um den wissenschafltichen Anspruch einzuhalten, müssen diese Quellen besonders kritisch refelektiert und mit anderen Publikationen verglichen werden.

\subsection{Validierung der Modelle}
Im vorherigen Kapitel wurden anhand von internationalen Studien Verfahren für das PCR-Pooling erarbeitet.
Die ermittelten Modelle sollen in diesem Kapitel durch Methoden aus der Informatik validiert werden.

Ziel ist es, eine geeignete Methode und Skalierung zu finden und diese auf Robustheit gegen Fehlern zu überprüfen.
In der Medizin sowie im betrieblichen Umfeld, funktioniert Skalierung grundsätzlich anders als in der Informatik.
Hier bedeutet die Verdopplung der Personenzahl eine massiven Mehraufwand bei Logistik und Organisation.
In diesem Kapitel sollen für das Modell optimale Parameter gefunden werden, um die in der Forschungsfrage formulierten Ziele zu erfüllen.
Dies soll die Grundlage der Auswahl eines effizienten Algorithmus für die Implementierung sein.

Als Forschungsmethoden die Validierung sind eine Simulation oder argumentativ-deduktiven Analyse vorgesehen.
Hierbei sollen Szenarien gefunden werden, in denen das Modell ein unerwünschtes Ergebnis liefert.
Auf Grundlage der Validierung werden gegebenenfalls Anpassungen an den Modellen vorgenommen.
Hierbei soll ein möglichst effizientes und robustes Modell entstehen und geprüft werden, ob Optimierungspotenzial gegenüber dem bisherigen Verfahren der Blutspendedienste existiert.

\section{Implementierung im betrieblichen Umfeld}
In diesem Kapitel soll eine Referenzimplementierung der erarbeiteten Modelle in das betriebliche Umfeld erstellt werden.
Menschliche Fehler können zu einem Versagen des Systems führen und sind durch betriebliche Abläufe zu minimieren.

Praktische Erwägungen wie Logistik und Kosten der Testung werden beschrieben.

\section{Ergebnis}
Die Arbeit endet mit einem Kapitel, in welchem die Erlebnisse zusammengefasst und Handlungsempfehlungen gegeben werden.
Es wird geprüft ob das Forschungsziel erreicht wurde und wie hoch das Optimierungspotenzial gegenüber den bisherigen Verfahren ist.
Auf dieser Basis wird ein Ausblick auf die Potenziale der betrieblichen Umsetzung gegeben.

